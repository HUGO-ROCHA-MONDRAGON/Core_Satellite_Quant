\documentclass[11pt,a4paper]{article}

% ── Packages ──
\usepackage[utf8]{inputenc}
\usepackage[T1]{fontenc}
\usepackage[french]{babel}
\usepackage{amsmath, amssymb, amsthm}
\usepackage{graphicx}
\usepackage{booktabs}
\usepackage{geometry}
\usepackage{hyperref}
\usepackage{xcolor}
\usepackage{float}
\usepackage{enumitem}
\usepackage{caption}
\usepackage{subcaption}
\usepackage{fancyhdr}
\usepackage{titlesec}

\geometry{margin=2.5cm}
\hypersetup{colorlinks=true, linkcolor=blue!60!black, citecolor=blue!60!black, urlcolor=blue!60!black}

% ── En-tête / pied de page ──
\pagestyle{fancy}
\fancyhf{}
\fancyhead[L]{\small Core-Satellite Quantitatif}
\fancyhead[R]{\small\thepage}
\fancyfoot[C]{\small \today}

% ── Titre ──
\title{\textbf{Construction d'un Portefeuille Core-Satellite} \\ 
\large Frontière Efficiente, Métriques de Concentration \\ et Backtest Rolling Trimestriel}
\author{}
\date{\today}

\begin{document}
\maketitle
\thispagestyle{fancy}

\begin{abstract}
Ce document présente la construction d'un portefeuille \textit{Core} composé de 3~ETFs couvrant les classes Equity, Rates et Credit. Partant de l'optimisation Moyenne-Variance classique (Markowitz), nous diagnostiquons le problème de concentration à l'aide de trois indicateurs complémentaires (HHI, Diversification Ratio, Max Risk Contribution), puis évaluons sept méthodes d'allocation alternatives. Un backtest rolling trimestriel valide les résultats en out-of-sample sur la période 2010--2025.
\end{abstract}

\tableofcontents
\newpage

% ══════════════════════════════════════════════════════════
\section{Données et sélection des ETFs}
% ══════════════════════════════════════════════════════════

\subsection{Source des données}

Les données proviennent d'un fichier Excel (\texttt{core\_ETF.xlsx}) contenant les prix journaliers d'un univers d'ETFs européens. Le fichier est structuré en colonnes paires (dates) et impaires (prix), avec trois lignes de métadonnées : tickers, buckets (classe d'actifs) et noms.

\subsection{Sélection quantitative}

Pour chaque thème d'investissement (Equity, Rates, Credit), on sélectionne l'ETF le plus représentatif via un score composite :
\begin{equation}
    \text{Score}_i = \bar{\rho}_i + 0{,}0001 \times n_{\text{obs},i} - 0{,}1 \times \sigma_i
\end{equation}
où $\bar{\rho}_i$ est la corrélation moyenne de l'ETF $i$ avec son bucket, $n_{\text{obs},i}$ la profondeur d'historique, et $\sigma_i$ sa volatilité journalière. Ce score équilibre représentativité, robustesse statistique et stabilité.

\subsection{ETFs retenus}

\begin{table}[H]
\centering
\caption{ETFs sélectionnés pour le portefeuille Core}
\label{tab:etfs}
\begin{tabular}{llll}
\toprule
\textbf{ETF} & \textbf{Classe} & \textbf{Bucket} & \textbf{Rôle} \\
\midrule
SWDA & Equity DM & Core & Moteur de rendement (actions mondiales) \\
CBE3 & Rates EMU & Govies Bucket & Couverture duration (obligations d'État EUR) \\
IEAC & Credit EMU & IG Core & Portage crédit investment-grade EUR \\
\bottomrule
\end{tabular}
\end{table}

La période d'analyse couvre \textbf{2010--2025} avec environ 4\,040 observations journalières. Les rendements sont calculés en log-rendements :
\begin{equation}
    r_t = \ln\!\left(\frac{P_t}{P_{t-1}}\right)
\end{equation}

Les paramètres annualisés sont estimés sur l'ensemble de la période :
\begin{equation}
    \mu = \bar{r} \times 252, \qquad \Sigma = \text{Cov}(r) \times 252
\end{equation}

% ══════════════════════════════════════════════════════════
\section{Frontière efficiente et optimisation Moyenne-Variance}
% ══════════════════════════════════════════════════════════

\subsection{Cadre théorique}

L'optimisation Moyenne-Variance (Markowitz, 1952) cherche le portefeuille minimisant la volatilité pour un rendement cible $\mu^*$ :
\begin{equation}
    \min_{w} \quad \sigma_p = \sqrt{w^T \Sigma\, w}
    \qquad \text{s.c.} \quad w^T \mu = \mu^*, \quad \sum_i w_i = 1, \quad w_i \geq 0
\end{equation}

En faisant varier $\mu^*$ sur l'intervalle des rendements atteignables, on trace la \textbf{frontière efficiente} : l'ensemble des portefeuilles offrant le meilleur compromis rendement/risque.

\subsection{Portefeuilles remarquables}

\begin{itemize}[nosep]
    \item \textbf{Min-Variance} : $\arg\min_w \sqrt{w^T \Sigma\, w}$ — point le plus à gauche de la frontière.
    \item \textbf{Max-Sharpe} (portefeuille tangent) : $\arg\max_w \frac{w^T\mu}{\sqrt{w^T \Sigma\, w}}$ — point de tangence avec la CML.
\end{itemize}

La \textbf{Capital Market Line} (CML) relie le taux sans risque $r_f = 0$ au portefeuille tangent :
\begin{equation}
    \mathbb{E}[r_p] = r_f + \frac{\mu_{\text{tan}} - r_f}{\sigma_{\text{tan}}} \cdot \sigma_p
\end{equation}

\subsection{Simulation Monte Carlo}

Pour visualiser l'espace des portefeuilles atteignables, nous simulons 15\,000 portefeuilles aléatoires via une distribution de Dirichlet :
\begin{equation}
    w \sim \text{Dir}(\mathbf{1}_N), \qquad \text{soit} \quad w_i \geq 0, \quad \sum_i w_i = 1
\end{equation}

Ce nuage de points, coloré par Sharpe Ratio, est superposé à la frontière efficiente exacte obtenue par optimisation.

\begin{figure}[H]
    \centering
    % \includegraphics[width=0.95\textwidth]{fig_frontiere_efficiente.png}
    \caption{Frontière efficiente — 3~ETFs Core (SWDA, CBE3, IEAC). Les 7~méthodes d'allocation sont positionnées sur le graphique. Le nuage Monte Carlo (15\,000 portefeuilles) est coloré par Sharpe Ratio.}
    \label{fig:frontier}
\end{figure}

% ══════════════════════════════════════════════════════════
\section{Diagnostic de concentration}
% ══════════════════════════════════════════════════════════

L'optimisation classique (Min-Variance, Max-Sharpe) produit des portefeuilles \textbf{très concentrés} : CBE3 reçoit plus de 85\% du poids. Trois indicateurs complémentaires permettent de mesurer cette concentration sous différents angles.

\subsection{HHI (Herfindahl-Hirschman Index)}

Le HHI mesure la concentration \textit{des poids} :
\begin{equation}
    HHI = \sum_{i=1}^{N} w_i^2
\end{equation}

\begin{itemize}[nosep]
    \item $HHI = 1/N \approx 0{,}333$ : portefeuille parfaitement équipondéré (diversification maximale des poids).
    \item $HHI \to 1$ : portefeuille quasi mono-actif.
\end{itemize}

\subsection{Diversification Ratio (DR)}

Le DR mesure l'exploitation des \textit{corrélations imparfaites} :
\begin{equation}
    DR = \frac{w^T \sigma}{\sigma_p} = \frac{\sum_i w_i \sigma_i}{\sqrt{w^T \Sigma\, w}}
\end{equation}

\begin{itemize}[nosep]
    \item $DR = 1$ : aucun gain de diversification (corrélation parfaite ou mono-actif).
    \item $DR > 1$ : le portefeuille bénéficie de la diversification. Plus $DR$ est élevé, mieux les corrélations sont exploitées.
\end{itemize}

\subsection{Max Risk Contribution (MaxRC)}

Le MaxRC mesure la concentration \textit{du risque} (et non des poids) :
\begin{equation}
    RC_i = w_i \cdot \frac{(\Sigma\, w)_i}{\sigma_p}, \qquad MaxRC = \max_i \frac{RC_i}{\sum_j RC_j}
\end{equation}

\begin{itemize}[nosep]
    \item $MaxRC = 1/N \approx 33\%$ : chaque actif contribue également au risque total.
    \item Un portefeuille peut avoir des poids équilibrés mais un risque dominé par un seul actif (ex. : Equal-Weight avec un actif très volatil).
\end{itemize}

\begin{figure}[H]
    \centering
    % \includegraphics[width=\textwidth]{fig_tradeoff_3metriques.png}
    \caption{Trade-off Sharpe Ratio vs trois indicateurs de concentration (HHI, Diversification Ratio, Max Risk Contribution). Le nuage gris représente les 15\,000 portefeuilles simulés.}
    \label{fig:tradeoff}
\end{figure}

% ══════════════════════════════════════════════════════════
\section{Sept méthodes d'allocation}
% ══════════════════════════════════════════════════════════

\subsection{Méthodes classiques (non contraintes)}

\paragraph{Min-Variance.}
\begin{equation}
    w^{\text{MV}} = \arg\min_w \sqrt{w^T \Sigma\, w} \qquad \text{s.c.} \quad \sum w_i = 1, \; w_i \geq 0
\end{equation}
Produit le portefeuille de volatilité minimale mais tend à concentrer sur l'actif le moins volatil.

\paragraph{Max-Sharpe (portefeuille tangent).}
\begin{equation}
    w^{\text{MS}} = \arg\max_w \frac{w^T \mu}{\sqrt{w^T \Sigma\, w}} \qquad \text{s.c.} \quad \sum w_i = 1, \; w_i \geq 0
\end{equation}
Meilleur Sharpe théorique mais très sensible aux erreurs d'estimation de $\mu$.

\subsection{Méthodes naïves}

\paragraph{Equal-Weight.}
\begin{equation}
    w_i^{\text{EW}} = \frac{1}{N} \quad \forall i
\end{equation}
Simple et robuste, HHI minimal par construction, mais ignore toute information de risque et rendement.

\paragraph{Inverse Volatility.}
\begin{equation}
    w_i^{\text{IV}} = \frac{1/\sigma_i}{\sum_j 1/\sigma_j}
\end{equation}
Surpondère les actifs moins volatils. Ignore les corrélations entre actifs.

\subsection{Méthodes à objectif de diversification}

\paragraph{Risk Parity.}
Cherche à égaliser la contribution au risque de chaque actif :
\begin{equation}
    w^{\text{RP}} = \arg\min_w \sum_{i=1}^{N} \left(RC_i - \frac{\sigma_p}{N}\right)^2 \qquad \text{s.c.} \quad \sum w_i = 1, \; w_i \geq 0{,}01
\end{equation}
Assure un $MaxRC$ faible mais peut s'éloigner de la frontière efficiente.

\paragraph{Max Diversification.}
\begin{equation}
    w^{\text{MD}} = \arg\max_w \frac{w^T \sigma}{\sqrt{w^T \Sigma\, w}} \qquad \text{s.c.} \quad \sum w_i = 1, \; w_i \geq 0{,}01
\end{equation}
Maximise le Diversification Ratio, exploitant au mieux les corrélations imparfaites.

\subsection{Méthode hybride}

\paragraph{Sharpe sous contrainte de poids (plafond 50\,\%).}
\begin{equation}
    w^{\text{SC}} = \arg\max_w \frac{w^T \mu}{\sqrt{w^T \Sigma\, w}} \qquad \text{s.c.} \quad \sum w_i = 1, \; 0{,}05 \leq w_i \leq 0{,}50
\end{equation}
Combine l'objectif de Sharpe maximal avec une contrainte de diversification minimale. Le plafond de 50\% empêche la concentration excessive tout en laissant à l'optimiseur une liberté suffisante.

\begin{figure}[H]
    \centering
    % \includegraphics[width=\textwidth]{fig_poids_metriques.png}
    \caption{Gauche : répartition des poids pour les 7~méthodes d'allocation. Droite : tableau des métriques comparées (Ret\%, Vol\%, Sharpe, HHI, DivRatio, MaxRC\%).}
    \label{fig:weights}
\end{figure}

% ══════════════════════════════════════════════════════════
\section{Résultats de l'analyse statique (full-sample)}
% ══════════════════════════════════════════════════════════

\subsection{Tableau comparatif}

\begin{table}[H]
\centering
\caption{Comparaison des 7~méthodes — analyse statique full-sample (2010--2025)}
\label{tab:static}
\begin{tabular}{l*{6}{c}}
\toprule
\textbf{Méthode} & \textbf{Ret\%} & \textbf{Vol\%} & \textbf{Sharpe} & \textbf{HHI} & \textbf{DR} & \textbf{MaxRC\%} \\
\midrule
Min-Variance       & 0,81  & 1,41  & 0,573 & 0,799 & 1,31 & 88,7 \\
Max-Sharpe         & 2,10  & 2,27  & 0,926 & 0,781 & 1,39 & 66,6 \\
Equal-Weight       & 4,12  & 5,33  & 0,774 & 0,333 & 1,28 & 88,6 \\
Risk Parity        & 1,78  & 2,80  & 0,637 & 0,408 & 1,51 & 54,2 \\
Max Diversif.      & 1,38  & 1,78  & 0,774 & 0,564 & 1,66 & 36,4 \\
Sharpe plaf.~50\%  & 3,98  & 4,87  & 0,818 & 0,383 & 1,27 & 92,8 \\
Inv.~Volatility    & 1,43  & 1,87  & 0,766 & 0,536 & 1,65 & 34,8 \\
\bottomrule
\end{tabular}
\end{table}

\subsection{Analyse du triangle d'incompatibilité}

Les résultats révèlent un \textbf{triangle d'incompatibilité} entre trois objectifs :

\begin{enumerate}[nosep]
    \item \textbf{Performance (Sharpe)} : Max-Sharpe domine (0,926) mais avec HHI = 0,781.
    \item \textbf{Diversification des poids (HHI)} : Equal-Weight minimise le HHI (0,333) mais le Sharpe est sous-optimal.
    \item \textbf{Diversification du risque (MaxRC, DR)} : Inv.~Volatility et Max~Diversif.\ obtiennent les meilleurs MaxRC ($\approx 35\%$) et DR ($\approx 1{,}65$).
\end{enumerate}

Les méthodes intermédiaires (Risk Parity, Sharpe plaf.~50\%) offrent le meilleur \textbf{compromis} entre ces trois dimensions.

% ══════════════════════════════════════════════════════════
\section{Backtest rolling trimestriel}
% ══════════════════════════════════════════════════════════

\subsection{Méthodologie}

L'analyse statique utilise les paramètres estimés sur toute la période, ce qui induit un \textit{look-ahead bias}. Le backtest rolling corrige ce biais :

\begin{enumerate}[nosep]
    \item \textbf{Fenêtre d'estimation} : $L = 252$ jours ouvrés (1~an glissant).
    \item \textbf{Rebalancement} : tous les $\Delta = 63$ jours ($\approx$ 1~trimestre).
    \item À chaque date de rebalancement $t_k$ :
    \begin{itemize}[nosep]
        \item On estime $\hat{\mu}_k$ et $\hat{\Sigma}_k$ sur la fenêtre $[t_k - L, \; t_k)$.
        \item On calcule les poids optimaux $w_k$ selon la méthode choisie.
        \item On applique $w_k$ au trimestre suivant $[t_k, \; t_k + \Delta)$ (\textit{out-of-sample}).
    \end{itemize}
\end{enumerate}

La performance cumulée OOS est construite comme :
\begin{equation}
    V_t = \prod_{s=1}^{t} \left(1 + w_{k(s)}^T r_s\right)
\end{equation}
où $k(s)$ est l'indice du dernier rebalancement avant la date $s$.

\subsection{Résultats OOS}

\begin{table}[H]
\centering
\caption{Performance out-of-sample — rolling trimestriel (2011--2025, 60~rebalancements)}
\label{tab:rolling}
\begin{tabular}{l*{7}{c}}
\toprule
\textbf{Méthode} & \textbf{Ret\%} & \textbf{Vol\%} & \textbf{Sharpe} & \textbf{MaxDD} & \textbf{HHI} & \textbf{DR} & \textbf{MaxRC\%} \\
\midrule
Sharpe plaf.~50\% & 3,53  & 5,61  & 0,630 & $-$11,6\% & 0,434 & 1,32 & 82,4 \\
Risk Parity        & 1,25  & 2,49  & 0,502 & $-$14,3\% & 0,454 & 1,56 & 50,8 \\
Equal-Weight       & 4,08  & 5,33  & 0,765 & $-$13,3\% & 0,333 & 1,26 & 90,1 \\
Inv.~Volatility    & 0,75  & 1,77  & 0,424 & $-$11,9\% & 0,588 & 1,69 & 36,4 \\
Max Diversif.      & 0,77  & 1,66  & 0,461 & $-$11,2\% & 0,610 & 1,70 & 37,2 \\
\bottomrule
\end{tabular}
\end{table}

\begin{figure}[H]
    \centering
    % \includegraphics[width=\textwidth]{fig_rolling_backtest.png}
    \caption{Backtest rolling trimestriel (2011--2025). Haut gauche : performance cumulée OOS. Haut droit : évolution des poids (Sharpe plaf.~50\%). Bas gauche : métriques de concentration dans le temps. Bas droit : tableau récapitulatif.}
    \label{fig:rolling}
\end{figure}

\subsection{Analyse des résultats OOS}

Le backtest sur 60~rebalancements confirme plusieurs enseignements :

\begin{itemize}[nosep]
    \item \textbf{Equal-Weight} obtient le meilleur Sharpe OOS (0,765), bénéficiant de l'exposition actions sur une période haussière (2010--2025).
    \item \textbf{Sharpe plaf.~50\%} offre un bon compromis avec un Sharpe de 0,630 et un drawdown modéré ($-$11,6\%).
    \item Les méthodes axées sur la diversification du risque (Inv.~Vol, Max~Diversif.) maintiennent un MaxRC très bas ($\approx 36$--$37\%$) mais au prix d'un rendement OOS faible.
    \item Le \textbf{Risk Parity} subit le plus gros drawdown ($-$14,3\%), surpondérant les deux branches obligataires qui ont souffert en 2022.
\end{itemize}

% ══════════════════════════════════════════════════════════
\section{Conclusion et recommandation}
% ══════════════════════════════════════════════════════════

L'approche Core-Satellite avec 3~ETFs et l'optimisation Moyenne-Variance classique présente un problème structurel de \textbf{concentration excessive} :\ les méthodes Min-Variance et Max-Sharpe allouent plus de 85\% à CBE3 (govies EUR), ce qui est confirmé par les trois indicateurs (HHI $> 0{,}78$, MaxRC $> 66\%$, DR $< 1{,}4$).

Sept méthodes d'allocation ont été comparées sur trois dimensions : performance (Sharpe), diversification des poids (HHI) et diversification du risque (DR, MaxRC). Les graphiques de trade-off montrent qu'aucune méthode ne domine simultanément sur les trois axes, révélant un \textbf{triangle d'incompatibilité}.

L'allocation \textbf{Sharpe plafonnée à 50\%} constitue le meilleur compromis :
\begin{itemize}[nosep]
    \item Sharpe statique compétitif (0,818), second après Max-Sharpe.
    \item HHI modéré (0,383) grâce à la contrainte de poids.
    \item Performance OOS robuste confirmée par le backtest rolling (Sharpe 0,630, MaxDD $-$11,6\%).
\end{itemize}

Le \textbf{Risk Parity} reste une alternative valide pour les investisseurs privilégiant la stabilité du risque ($MaxRC \approx 50\%$, $DR \approx 1{,}56$), au prix d'un rendement plus faible.

\vspace{1em}
\noindent\textit{``La diversification est la seule chose gratuite en finance.''} — Harry Markowitz

\end{document}
